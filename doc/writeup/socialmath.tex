\documentclass{acm_proc_article-sp}
%\usepackage{usenix,epsfig,endnotes}
\begin{document}

%don't want date printed
\date{}

%make title bold and 14 pt font (Latex default is non-bold, 16 pt)
\title{\Large \bf Social Dynamics Used for Verification of Large Proof Trees.}

\author{
{\rm Jianchi Chen}\and
{\rm Ying-Yu Ho}\and
{\rm Tim Holland}\and
{\rm Kexin Rong}
{\em California Institute of Technology}
}

\maketitle

% Use the following at camera-ready time to suppress page numbers.
% Comment it out when you first submit the paper for review.
\thispagestyle{empty}

\subsection*{Abstract}
A mathematical proof has the structure of a DAG. However, it is typically
presented in a linear fashion, and cannot practically expose the depth of its
roots. We propose a web service to provide both of these by exposing part of
the graph to the user and allowing them to explore down into the subproofs.
Naturally it would be prohibitive for the authors to write even a useful amount
of the tree in, so the service allows the introduction of new proofs to expand
the reach of the proof trees. Specification of proofs could either be very
formal, in which case a program could verify them or more relaxed and reviewed
by other users in a social effort to find mathematical truth.

\section{Introduction}
We propose a web application that allows the inspection of, addition to, and
potentially input of verification of collection of large proof DAGs.
Interfacing with a node of the tree will present a statement of the theorem
associated with that node, edges to the children and potentially to some of its
parents. Text accompanies the node, and explains how the children are combined
together to prove the theorem. Ideally the accompanying text is as small as
possible and the nodes have a small fan out to break the proof down into
digestible chunks.

We have two possible choices to proceed on how to keep the DAG logically sound.
The first is more of a technical problem: require highly precise statements of
additions to the tree to be parsed, the logical statements can then be verified
(or rejected) automatically. Alternatively, with a more social emphasis proofs
could be stated in natural mathematical language. Under this scheme, the
correctness would have to be verified by users, through a combination
of moderation, reputation, and democracy.

\section{Technical Hurdles}
At the very basic level we have to deal with all the usual problems in
building a webapp: deciding upon tools (languages, databases, frameworks)
and an API to communicate between client and server. Further features
needed would be dynamic loading of the graph as the user scrolls through,
search to look for other proofs to minimize redundancy, automated theorem
verification or a voting mechanism, ability to mention other theorems
to include them is lemmas in a particular proof, text entry for a theorem,
inclusion of images for proofs that necessitate it, and collections of
axioms for the various fields of mathematics that could be present as
a starting point. Hard problems would involve the choice of data 
structure for the proofs, intuitive presentation of the DAG to the user,
an effective mechanism of rejecting poorly worded or incorrect proofs.

\section{Potential Order of Implementation}
\begin{enumerate}
    \item Persistent storage of a proof in the database.
    \item Communication of a proof over the network.
    \item User interface.
    \item Adding a new theorem from the user interface.
    \item Persistence of theorems in UI across page loads.
    \item Verification system (social or formal). 
\end{enumerate}

\section{Background}
I'm not aware of any similar projects.
% \bibliography{sample}}
% \theendnotes
\section{End Product}
The end product would hopefully be a useful web application that would
allow publishing of proofs in a graphical format subject to immediate review
of other users, and a repository for a large collection of proofs to be
seen in a graphical format upon request.
\end{document}
