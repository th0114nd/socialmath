\documentclass{acm_proc_article-sp}
%\usepackage{usenix,epsfig,endnotes}
\begin{document}


%make title bold and 14 pt font (Latex default is non-bold, 16 pt)
\title{\Large \bf Social Dynamics Used for Verification of Large Proof Trees.}
\numberofauthors{4} %  in this sample file, there are a *total*
% of EIGHT authors. SIX appear on the 'first-page' (for formatting
% reasons) and the remaining two appear in the \additionalauthors section.
%
\author{
\alignauthor
Jianchi Chen\\
       \affaddr{California Institute of Technology}\\
       \affaddr{1200 E. California Blvd.}\\
       \affaddr{Pasadena, United States}\\
       \email{jchen2@caltech.edu}
\alignauthor
Ying-Yu Ho\\
       \affaddr{California Institute of Technology}\\
       \affaddr{1200 E. California Blvd.}\\
       \affaddr{Pasadena, United States}\\
       \email{yingyu@caltech.edu}
\alignauthor
Tim Holland\\
       \affaddr{California Institute of Technology}\\
       \affaddr{1200 E. California Blvd.}\\
       \affaddr{Pasadena, United States}\\
       \email{th0114nd@gmail.com}
\and  % use '\and' if you need 'another row' of author names
\alignauthor 
Kexin Rong\\
        \affaddr{California Institute of Technology}\\
       \affaddr{1200 E. California Blvd.}\\
       \affaddr{Pasadena, United States}\\
       \email{krong@caltech.edu}
}

\maketitle

% Use the following at camera-ready time to suppress page numbers.
% Comment it out when you first submit the paper for review.
\thispagestyle{empty}

\subsection*{Abstract}
A mathematical proof has the structure of a DAG. However, it is typically
presented in a linear fashion, and cannot practically expose the depth of its
roots. We propose a web service to provide both of these by exposing part of
the graph to the user and allowing them to explore down into the subproofs.
Naturally it would be prohibitive for the authors to write even a useful amount
of the tree in, so the service allows the introduction of new proofs to expand
the reach of the proof trees. Specification of proofs could either be very
formal, in which case a program could verify them or more relaxed and reviewed
by other users in a social effort to find mathematical truth.

\section{Introduction}
We propose a web application that allows the inspection of, addition to, and
potentially input of verification of collection of large proof DAGs.
Interfacing with a node of the tree will present a statement of the theorem
associated with that node, edges to the children and potentially to some of its
parents. Text accompanies the node, and explains how the children are combined
together to prove the theorem. Ideally the accompanying text is as small as
possible and the nodes have a small fan out to break the proof down into
digestible chunks.

We have two possible choices to proceed on how to keep the DAG logically sound.
The first is more of a technical problem: require highly precise statements of
additions to the tree to be parsed, the logical statements can then be verified
(or rejected) automatically. Alternatively, with a more social emphasis proofs
could be stated in natural mathematical language. Under this scheme, the
correctness would have to be verified by users, through a combination
of moderation, reputation, and democracy.

\section{Technical Hurdles}
At the very basic level we have to deal with all the usual problems in
building a webapp: deciding upon tools (languages, databases, frameworks)
and an API to communicate between client and server. Further features
needed would be dynamic loading of the graph as the user scrolls through,
search to look for other proofs to minimize redundancy, automated theorem
verification or a voting mechanism, ability to mention other theorems
to include them is lemmas in a particular proof, text entry for a theorem,
inclusion of images for proofs that necessitate it, and collections of
axioms for the various fields of mathematics that could be present as
a starting point. Hard problems would involve the choice of data 
structure for the proofs, intuitive presentation of the DAG to the user,
an effective mechanism of rejecting poorly worded or incorrect proofs.

\section{Potential Order of Implementation}
\begin{enumerate}
    \item Persistent storage of a proof in the database.
    \item Communication of a proof over the network.
    \item User interface.
    \item Adding a new theorem from the user interface.
    \item Persistence of theorems in UI across page loads.
    \item Verification system (social or formal). 
\end{enumerate}

\section{Background}
I'm not aware of any similar projects.
% \bibliography{sample}}
% \theendnotes

\section{Timeline}
\subsection{Initial Timeline}
Below is the initial project timeline we proposed at the project proposal presentation. The plan was to build a working prototype before midterm, and to add as many features as time allows afterwards.  
\begin{itemize}
\item Week 2 - 5: Basics
\begin{itemize}
\item Client-server API
\item Database API
\item Implement views and templates for browsing, editing, and adding theorems
\item Front end visualization
\end{itemize}
\item Week 6: Milestone:\\
Should have a simple working prototype, which allows adding, deleting and viewing nodes from the proof DAG
\item Week 7 - 9: Useful advanced features
\begin{itemize}
\item Search by keywords
\item User profile
\item private graph
\item Auto-keyword-extraction
\item Suggested theorems
\end{itemize}
\end{itemize}

\subsection{Actual Timeline}
Below is a summary of our actual weekly progress based on the blogs. The front end progress is highly non-linear because we either underestimated the amount of work or overestimated the ability to stick to schedule of the person in charge. Thus many features were first implemented on server-side and enhanced with Ajax later. Nevertheless, we were able to implement the three most useful advanced features: keyword search, user profile and private graph. A detail description of the end product is given in the next section.
\begin{enumerate}
\item \emph{Week 1: Changing Gears}\\\\
Due to the discovery of a fairly comprehensive open source traffic simulator project, we decided to abandon the original project proposal and to work, instead, on a web app to accommodate the visualization of mathematical proof tree structures. We discussed the basic application architecture: the typical LAMP stack with a REST API and CRUD functionality. In our case the P stands for Python	with Django REST framework, the M stands for SQLite, and the A could either be Apache or Amazon Web Service.\\\\
The first steps we took were similar to any team making a CRUD app: setting up the database and a method of communication between the client and server. Specifically, we determined a schema for the database and fleshed out data structures for the DAG. We also started to familiarized ourselves with the frameworks that we will be working with for the rest of the term.\\\\
On the front end side, several JavaScript frameworks quickly came to our mind:
AngularJS and/or jQuery for DOM manipulation, Bootstrap for style and UI components, and D3.js for data visualization. These frameworks have some overlapping functions so it was important to use them in a way that avoided conflicts and, preferably, reflected separation of concerns. Since we had little experience in web development (actually the main front-end architect had none), it would take significant time to research on JavaScript and its frameworks before making design decisions.\\\\

\item \emph{Week 2: Getting Started }\\\\
This week the first draft of the API was written. It's a fairly
standard one that supports the collection of all entries in
a paginated manner, getting a single entry for a close up view,
and the CRUD operations expected to have. \\\\
On the server side, we've decided to go for AWS instead of Apache. As a result, we switched to using virtualenv, which gave us consistent versioning across our local machines as well as allowing us to use versions required by AWS without having conflicts of the development computers. We also migrated from SQLite to MySQL, since AWS had the best support for the latter. Although setting up MySQL was slightly frustrating.\\\\
On the backend side, we implemented serializers to help form JSON responses, and view methods for creating and reading theorems/proofs.\\\\
On the front end side, we started by finding data visualization examples to understand the power of available tools. One of our particular interest was \textbf{d3 process map} (https://github.com/nylen/d3-process-map), which demonstrated a hierarchical layout of graph that we would eventually mimic. Meanwhile, we realized that AngularJS, jQuery, and D3.js all provided DOM manipulation methods, which had the most potential of conflicts. Hence we decided to structure front end applications on MVC (model-viewer-controller) framework of AngularJS, avoid jQuery as possible, and use D3.js mainly as a graph layout engine.\\\\
No JavaScript codes were written yet, but we were able to write some Django view templates to provide text-based interaction with DAG.
\\

\item \emph{Week 3: Backend Milestone}\\\\
After adding templates and view methods for edit and deletion, we had a fully working set of CRUD functionalities implemented this week. We also did substantial unit tests on database models and api calls to make sure that our backend was robust so far. \\\\
At this point, the backend has already reached the original midterm milestone. In the front end, we succeeded in driving AngularJS templates with D3 graph layout engine. However, time was wasted on studying hierarchical force layout algorithm of d3 process map. While force layout was provided by D3.js, it turned out that the hierarchical part was manually specified in the dataset, so we needed to invent our own algorithm. A minor issue was that we formatted graphs as adjacency lists while D3 Force Layout used nodes and edges as native objects, which required some conversion.\\

\item \emph{Week 4: Moving on}\\\\
This week, the backend side has moved on to advanced features, starting with the keyword related ones. We implemented keyword creation, edit and search. Users can add an arbitrary number of keywords for theorems/proofs, and a search bar was added on the index page to allow search by keywords.\\\\
As we moved one person from the backend to the front end, the front end side has made substantial progress this week. First, we wrote a stylesheet for the website to make it more user friendly. We also set up MathJax, a free online LaTex compiling tool, to support adding and displaying LaTex code on the website. Finally, we had a working dynamic knowledge graph layout algorithm. However, the visualization was working as an independent component, and needed to be merged with the rest of the project.\\\\
\\

\item \emph{Week 5:  MVP online}\\\\
This week, we have been focusing on putting a simple working version of our product online. On one hand, we tried to connect the knowledge graph visualization with backend database. On the other hand, we worked on setting up Amazon EC2 (Amazon Elastic Compute Cloud) to host the website. We were able to seed the database with number theory theorems and deploy the latest backend code. The knowledge graph visualization was still unavailable online at this point. \\\\
Meanwhile, we have also been extending the website's functionalities. We started to implement user authentication system in the hope of enabling users to set up private knowledge graphs for classroom setttings.\\\\


\item \emph{Week 6: }\\



\item \emph{Week 7:  }\\

\item \emph{Week 8: }\\

\end{enumerate}
\section{End Product}
The end product would hopefully be a useful web application that would
allow publishing of proofs in a graphical format subject to immediate review
of other users, and a repository for a large collection of proofs to be
seen in a graphical format upon request.
\subsection{Front End}
\subsection{Back End}

\end{document}
